\chapter{FP Multiplier}

\section{Model Verification}

Before starting any work on the multiplier, we verified that it was working as intended.
Hence, we have created a file called \verb|tb_fpumult.vhd|, where the DUT computes the
square of the numbers stored in \verb|fp_samples.hex| and the result is compared against the
values stored in \verb|fp_prod.hex|.
Then, we added the required input registers and we have verified again that the results were correct.

\section{Synthesis}

We have performed various synthesis to analyze the differences between various implementations and constraints.
To do this, we have written three different synthesis scripts, that are \verb|syn_script.tcl|, \verb|syn_script_csa.tcl|
and \verb|syn_script_pparch.tcl|. The first one synthesizes the multiplier by leaving all the implementation choices to
the Synopsys' tool, while the second forces the usage of CSA-based multipliers and the last one the usage of parallel-prefix based
multipliers. The asked results are shown in table \ref{tab:ex1}.

\begin{center}
    \label{tab:ex1}
    \begin{tabular}{ |c|c|c| }
        \hline
        Multiplier architecture & $T_{ck}$ & Area \\
        \hline
        Chosen by the synthesizer & $1.6\ ns$ & $3999.04\ \mu m^2$ \\
        \hline
        CSA & $4.6\ ns$ & $4807.68\ \mu m^2$ \\
        \hline
        PPArch & $4.5\ ns$ & $3734.91\ \mu m^2$ \\
        \hline
    \end{tabular}
\end{center}

Performance is remarkable when the synthesizer is allowed to choose the implementations by itself. By looking at
the resources' report the synthesizer chooses the parallel prefix multiplier, although it is optimized differently
in respect to the one used in the synthesis done with \verb|syn_script_pparch.tcl|. In fact, the former is optimized
for both area and speed, while the latter only for area. However, for an increase of about only $7\%$ in area we obtain
a performance boost of $65\%$, hence in a real application there is no doubt in the implementation we would choose.

\section{Fine-grain Pipelining and Optimization}

We added, as asked, the register after the significands' multiplier in the second stage, as well as all the required registers
to maintain the correct timing. We verified the correctness of the updated design with the \verb|tb_fpumult_reg.vhd| testbench.
Then we have run two synthesis, one with the \verb|compile_ultra| command (\verb|syn_script_comp_ultra.tcl|) and one with a simple
\verb|compile| and the \verb|optimize_registers| (\verb|syn_script_opt_reg.tcl|). The results are summarized in table \ref{tab:ex1.1}.

\begin{center}
    \label{tab:ex1.1}
    \begin{tabular}{ |c|c|c| }
        \hline
        Synthesis commands & $T_{ck}$ & Area \\
        \hline
        \verb|compile + optimize_register| & $0.8\ ns$ & $4969.41\ \mu m^2$ \\
        \hline
        \verb|compile_ultra| & $1.5\ ns$ & $4216.1\ \mu m^2$ \\
        \hline
    \end{tabular}
\end{center}

The synthesis with retiming reaches double the frequency of both the one done with the \verb|compile| in the previous section and the one
done with the \verb|compile_ultra|. This shows the value of the optimization techniques we have studied. However, it suffers of an area
increase with respect to the \verb|compile_ultra| synthesis, since it probably adds more registers due to the non-negative register count
on the graph's arcs. Nevertheless, its speed is outstanding.