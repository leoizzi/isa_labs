\chapter{Testing}

\section{Adder}
To gain confidence with the UVM environment, we have run the simulation with the provided files first.
The resulting transcript is \verb|adder_transcript|, that shows the correctness of the circuit.

Then, we have modified \verb|packet_in.sv| to support input constraints. We have used the suggested
ranges for the operands, which are $100 \leq A \leq 1000$ and $B < 10*A$. The transcript related to this
simulation is called \verb|constraint_transcript|. This lesson has been very useful for the testing
of the multipliers.

As a final step, we have changed the reference model described in \verb|refmod.sv| to verify what happens
when QuestaSim detects a mismatch in the result. Specifically, we have turned the golden model adder into a
subtractor. The results of this simulation are stored in \verb|mismatch_transcript|. As expected, there are
only mismatches. This can be visualized at the end of the transcript, where the UVM Report Summary is written.

\section{MBE-Dadda tree multiplier}
To test the multiplier we used the adder infrastructure as a starting point and then we changed few files.
The first step was to add all the VHDL files developed for the lab2, which are \verb|ha.vhd|, \verb|fa.vhd|,
\verb|dadda.vhd|, \verb|mbe.vhd| and \verb|mult.vhd|.

The first modification has been done in \verb|DUT.sv|. Here, we have replaced the adder's instantiation with
the multiplier's one. This change is not enough though, because the result of the adder fits in 32 bits, while
the multiplier's one fits in 64. Hence, we changed the size of \verb|data| in \verb|dut_if.sv| from 32 bits to
64 bits. This is all what concerns the changes in \verb|src/|.

In the \verb|tb/| folder, we have changed the reference model from an adder to a multiplier. Then, we have added
two constraints in \verb|packet_in.sv|, which are $A \geq 0$ and $B \geq 0$. This step was mandatory, because
the multiplier works only on unsigned numbers. Moreover, we have changed the output's data type of \verb|packet_out.sv|
from an integer to a longint.

Since the multiplier is written in VHDL while the testbench is written in System Verilog, we have solved the
integration by compiling first our multiplier and then the testbench. The result of the simulation is stored
in \verb|transcript_mult|. As expected, there were no mismatches.

\section{Floating point multiplier}
We have used the same strategy of the integer multiplier to have a baseline test infrastructure. We have then
imported in \verb|src/| all the required files. Since the FP multiplier is a sequential circuit, it was not enough
to change the circuit instantiation in \verb|DUT.sv|, but we needed to change also the FSM used by the testbench.
In particular, we have declared an integer variable called \verb|count| to count in the \verb|WAIT| state the
iterations. We remain in the \verb|WAIT| state until \verb|count == 6|, because after 5 cycles the circuit has
produced the final output. We've also sampled the input values when \verb|count == 0| due to a print mismatch
(which had no influences on the results).

In \verb|packet_in.sv| we generate the random numbers as integers for convenience, but we have added few
constraints to avoid mismatches due to the different handling of NaN, infinities and denormal numbers.
The checks are:

\begin{itemize}
    \item $(((A \& 32'h7f800000)>>23) + ((B \& 32'h7f800000)>>23) - 127) > 0$
    \item $(((A \& 32'h7f800000)>>23) + ((B \& 32'h7f800000)>>23) - 127) < 253$
    \item $((A \& 32'h7f800000)>>23) > 0$
    \item $((A \& 32'h7f800000)>>23) < 255$
    \item $((B \& 32'h7f800000)>>23) > 0$
    \item $((B \& 32'h7f800000)>>23) < 255$
\end{itemize}

The bit shifts and AND operations are used to extract the exponents from A and B. In fact, the numbers passed
to the reference model and the DUT are nothing more than the binary representations of these number.
The first two checks are used to avoid denormal numbers and overflows in the product. In the second check
we have considered the fact that the exponent may increase due to the rounding and normalizations.
The other checks are used to avoid special numbers in the inputs.

In \verb|refmod.sv| we convert the integers into shortreal by using the directive \verb|$bitstoshortreal|,
then we perform the floating point multiplication, and finally we store the final result as an integer through
the \verb|shortrealtobits| directive.

The result of the simulation is stored in \verb|transcript|.

In \verb|sim/| a script that can be used to launch the simulation in QuestaSim is also contained.