\chapter{Processor architecture}

In this chapter we will describe how we have implemented the processor. It is based on the structure studied during the
lectures. At a glance, it has 5 stages, the control unit is hardwired and it does not support forwarding.

\section{IF stage}
In this stage new instructions to be executed are fetched from the ROM. Its structure is pretty simple:
there is the program counter register (PC) that is updated at every clock cycle either with the target address,
that is the address of a jump, or with $PC + 4$. In case of a stall, the enable signal of the register is
de-asserted.

The current PC value is used to access a small ROM which is used only during simulations as it cannot be synthesized with
the processor. The current PC value, $PC + 4$ and the ROM output are fed to the IF/ID registers.

\section{ID stage}
Here the decode of the instruction fetched in the previous cycle is performed. The register file, as well as the immediate generator,
are in this stage. The register file is accessed regardless of the instruction type, and the control unit decides how
the immediate should be extended. These values, along with the instruction PC, its subsequent value and the destination register address, are propagated to the
ID/EXE registers.

The register file is capable of supporting two read and a single write per clock cycle. The write operation is done by the
WB stage on the falling edge of the clock signal. This means that it is possible to reduce the number of stalls by one,
because when an instruction writes in the RF from the WB stage its result can be immediately accessed by the instruction in the ID stage.

To allow the control unit to check whether there is the need for a stall, the source registers are sent to it.

\section{EXE stage}
The ALU supports two inputs, A and B. Based on the instruction, two muxes select which values have to be taken as operands.
For the A port, the choice is between the $PC$, $PC+4$, $0$ or the value coming from the RF's port 1. For the B port,
the choice is between the immediate, the RF's port 2 or $0$. In the ALU, all the functional units perform the computation in parallel,
then through a mux the result is selected. The functional units present in the ALU are:

\begin{itemize}
    \item the adder;
    \item the shifter;
    \item the logicals (AND and XOR);
    \item the set-less-than;
    \item the absolute value (only for \verb|risc_abs|).
\end{itemize}

Moreover, there is also a comparator that checks if the result is 0. In the affirmative case, it asserts the \verb|zero| signal.

To support branches and jumps there is a second adder, that takes as inputs the $PC$ and the immediate value shifted to the left
by one. This value represent the target address of the IF stage. However, it is selected by the IF stage only if two conditions
are true: in the control word the bit \verb|JMP_ENABLE| is set to 1, and if

\begin{itemize}
    \item the instruction is a \verb|jal| (detected by looking at the mux control signals, that is $A = NPC$ and $B = 0$)
    \item the instruction is a \verb|beq|, hence if the \verb|zero| signal is asserted.
\end{itemize}

Finally, to support \verb|store| operations, there is a direct connection between the second register port output and the EXE/MEM registers.

\section{MEM stage}
All instructions but \verb|lw| and \verb|sw| bypass the memory and their data is directly propagated to the MEM/WB registers.
The two aforementioned instructions, instead, are able to respectively read or write the RAM, which as for the ROM is not
synthesized but only simulated.

The RAM has a port dedicated to read operations and one dedicated to write operations, but only one of them can be executed
at any time. This has been done only to simplify the testing, since in this way there is no risk to misuse the tristate transistors.
To distinguish between read and write operations a dedicated \verb|wr| signal has been added.

Since the given test file assumes that the memory has been preloaded with the \verb|.data| segment, this RAM reads on reset
its initialization content from a file.

\section{WB stage}
In this stage there is simply a mux that, based on the instruction, selects if in the RF should be written the data coming
either from the ALU or the memory. The CU keeps track of whether the instruction is allowed to write content in the RF or not,
since the RF has an additional signal that enables the write operation.

\section{Control unit}
The control unit is hardwired, since for each instruction read in the IF stage it loads all the control signals inside the registers
and outputs them at the right time. This is achieved by pipelining the control word in the same way as the datapath.

In the ID stage it detects possible data hazards by looking at the data source addresses and the destination address in the ID/EXE and
EXE/MEM pipelining registers. If an hazard is detected, the control unit stalls the processor by inserting NOPs in the proper stages.

The control unit is also notified whenever the datapath is about to jump, because it needs to flush the IF and ID stages, as their
work is not needed anymore.
